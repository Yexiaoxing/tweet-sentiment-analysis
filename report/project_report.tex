\documentclass[12pt,a4paper]{article}

% ------------------------------------------------------------------------
% Packages
% ------------------------------------------------------------------------
\usepackage[body={7.2in, 10in},left=0.8in,right=0.8in]{geometry}
\usepackage{amsmath,amssymb,amsfonts,graphicx,amsthm,nicefrac,mathtools, verbatim}
\usepackage{amsmath,amssymb,amsthm}
\usepackage{enumitem}
\usepackage[numbers]{natbib}
\usepackage{hyperref}
\usepackage{breqn}
\usepackage{xcolor}
\usepackage{fancyhdr}
\usepackage{float}
\usepackage{multirow}

%% If you want to define a new command, you can do it like this:
\newcommand{\Q}{\mathbb{Q}}
\newcommand{\R}{\mathbb{R}}
\newcommand{\Z}{\mathbb{Z}}
\newcommand{\C}{\mathbb{C}}
\newcommand{\E}{\mathbb{E}}
\newcommand{\V}{\mathbb{V}}



\begin{document}
% ------------------------------------------------------------------------
% Course info
% ------------------------------------------------------------------------
\begin{center}
\textsc{ERG3020 Twitter} \\
CUHK-SZ, Apr 22, 2018
\\[\baselineskip]
		
% ------------------------------------------------------------------------
% Problem set info.   Remember to change the problem set number and student name!
% ------------------------------------------------------------------------
\textbf{\underline{Course Project}}
\end{center}
	
\noindent\textbf{Student Name:}  Ye Xiaoxing, Piao Chuxin

\noindent\textbf{Student ID:}  115010270, 115010058
	
\noindent\hrulefill


\section*{1. Abstract}

      Twitter is a great firehouse of real-time information. Using the data from twitter, people can gain information such as political leaning and other tendencies. Through the classification process of the data before the election day, the more competitive candidate with a higher approval rating can be obtained.



\section*{2. Introduction}

     American President Election’ is one of the catchiest affairs all over the world. To predict the candidate with higher approval rate beforehand, social networks can be effective in obtaining the political leaning of citizens in the United States. 

	Twitter is one of the influential social networks that can be used in predicting the political leaning of people who use twitter. Having downloaded the data of one day from the twitter website, the emotion scores have been computed, represented by a number from -1 to 1, in which 1 represents the positive emotion while -1 represents the negative emotions. After obtaining the score of emotion, the keywords which could represent different candidate have been searched. For example, ‘Trump’or‘ Donald’could represent Donald Trump, and ‘Hillary’or‘Clinton’could represent Hillary Clinton. Compared the twitter score which contains different key words to the average emotional score of all the twitter score, the rough political leaning of different candidates can be obtained.

\section*{3. Model}
\textbf{\large 3.1 }\\



\vspace{0.4cm}
\noindent\textbf{\large 3.2 }\\



\vspace{0.4cm}
\noindent\textbf{\large 3.3 }\\



\section*{4. Conclusion and Findings}


\end{document}